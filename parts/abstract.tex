Neutron stars are complex physical objects often modelled as a hydrodynamical system. Within these models, the two parameters energy density and pressure are related by an equation known as the ``equation of state''. This relationship allows us to calculate energy density if we know pressure, and visa versa. The equation of state itself is important because it encodes the information about interactions between the fundamental particles within the star. Furthermore, as neutron stars are extreme examples of gravitation, and not microscopically observable on Earth, the physics within neutron stars are still not well known. By postulating and analyzing different interactions within a neutron star, and therefore different equations of state, we can simulate the observable characteristics of the resulting star and compare them to empirical data to help better understand how neutron stars behave on a fundamental level.