\chapter{Quantum Hadrodynamics}

This chapter will describe the derivation of an equation of state from the QHD-I parameter set, as described within \autocite{diener_2008}.

\section{Introduction}


\section{Derivation of equations of motion}

The Euler-Lagrange equations, for a Lagrange density $\Lag$ over a classical field $\varphi_\alpha$, are given by
\begin{align}\label{eqn: ELE}
    \p_\nu\qty(\pdv{\Lag}{(\p_\nu \varphi_\alpha)}) - \pdv{\Lag}{\varphi_\alpha} = 0.
\end{align}
From \autocite[p. 56]{diener_2008}, we have the Lagrangian density of QHD-I
\begin{align} \label{eqn: QH, LD,QHD1}
    \Lag & = \bar{\psi} \bqty{\gamma_\mu(i\p^\mu -g_\nu V^\mu) - (M-g_s\phi)} \psi \nonumber\\
    & \quad + \frac{1}{2} \pqty{\p_\mu \phi \p^\mu \phi - m_s^2 \phi^2} - \frac{1}{4} V_{\mu\nu}V^{\mu\nu} + \frac{1}{2} m_\omega^2 V_\mu V^\mu,
\end{align}
where $V_{\mu\nu}\Def\p_\mu V_\nu (x) - \p^\nu V_\mu(x)$. To determine the equations of motion for this system, we must apply \eqref{eqn: ELE} to \eqref{eqn: QH, LD,QHD1} for each unique field in the system
\begin{align*}
    \varphi_\alpha = \begin{cases}
        \phi(x) &: \quad \text{scalar meson field,}\\
        V^\mu(x) &: \quad \text{vector meson field,}\\
        \psi(x) &: \quad \text{baryon field,}\\
        \bar\psi(x) &: \quad \text{Dirac adjoint baryon field,}\\
    \end{cases}
\end{align*}
where $\bar\psi(x) \Def \psi^\dagger(x)\gamma^0$, the \emph{Dirac adjoint.}

For the scalar meson field, when $\varphi_\alpha = \phi(x)$, the first term in \eqref{eqn: ELE} gives
\begin{align*}
    \p_\nu \pqty{\pdv{\Lag}{(\p_\nu \phi)}} = \frac{1}{2} \bqty{\pdv{(\p_\nu \phi)}\pqty{\p_\mu \p^\mu \phi}} = \p_\nu \p^\nu \phi,
\end{align*}
while the second term gives
\begin{align*}
    \pdv{\Lag}{\phi} = \bar\psi[+g_s]\psi + \frac{1}{2}\pqty{-m_s(2\phi)} = g_s \bar\psi \psi - m_s\phi.
\end{align*}
Combining, we get the first equation of motion,
\begin{align}\label{eqn: eom,sm}
    \p_\nu \pqty{\pdv{\Lag}{(\p_\nu \phi)}} - \pdv{\Lag}{\phi} = \p_\nu \p^\nu \phi - (g_s\bar\psi \psi - m_s\phi) = 0 \goesto \p_\nu \p^\nu \phi + m_s \phi = g_s \bar\psi \psi.
\end{align}
This is the form given in \autocite{diener_2008} (8.1a).

For the vector meson field, when $\varphi_\alpha = V_\mu$, the first term in \eqref{eqn: ELE} gives
\begin{align*}
    \p_\nu \pqty{\pdv{\Lag}{(\p_\nu V_\mu)}} = \p_\nu V^{\mu\nu},
\end{align*}
after many simplifications, including using the definition of $V^{\mu\nu}$ and the relabelling of indices. The second term gives
\begin{align*}
    \pdv{\Lag}{V_\mu} = \pdv{}{V_\mu}\bar\psi\bqty{\gamma_\alpha(-g_v V^\alpha) + \frac{1}{2} m_\omega^2 V^\alpha V_\alpha} = -g_v \bar\psi \gamma^\mu \psi + m_\omega^2 V^\mu,
\end{align*}
and combining we have
\begin{align}\label{eqn: vEOM, intermediate}
    \p_\nu \pqty{\pdv{\Lag}{(\p_\nu V_\mu)}} - \pdv{\Lag}{V_\mu} = \p_\nu V^{\mu\nu} - (-g_v \bar\psi \gamma^\mu \psi + m_\omega^2 V^\mu) = 0.
\end{align}
However, this is not the form given in \autocite{diener_2008}; to reach his form, we leverage the anti-symmetry of $V_{\mu\nu}$, namely
\begin{align*}
    V_{\mu\nu} = - V_{\nu\mu}.
\end{align*}
Therefore, we make the above substitution, multiply \eqref{eqn: vEOM, intermediate} by $-1$, and send $\mu \leftrightarrow \nu$ to obtain
\begin{align}\label{eqn: eom,vm}
    \p_\mu V^{\mu\nu} + m_\omega^2 V^\nu = g_v \bar\psi \gamma^\nu \psi,
\end{align}
as given in \cite{diener_2008} (8.1b).

Next, we have the two equations of motion from the baryon field. For $\varphi_\alpha = \bar\psi$, applying \eqref{eqn: ELE} is straight forward, as there is no $\p_\nu \bar\psi$ dependence in $\Lag$, so the first term in \eqref{eqn: ELE} is zero. Thus, we obtain
\begin{align}\label{eqn: eom,barpsi}
    \p_\nu \pqty{\pdv{\Lag}{(\p_\nu \bar\psi)}}  - \pdv{\Lag}{\bar\psi} = \bqty{\gamma_\mu(i\p^\mu - g_v V^\mu) - (M-g_s\phi)} \psi = 0.
\end{align}
For the final case, when $\varphi_\alpha = \psi$, the first term in \eqref{eqn: ELE} gives
\begin{align*}
    \p_\nu \pqty{\pdv{\Lag}{(\p_\nu \psi)}} = \p_\nu\bqty{\pdv{}{(\p_\nu\psi)}\bar\psi i \gamma_\alpha \p^\alpha \psi} = i\p_\nu \bar\psi \gamma^\nu,
\end{align*}
while the second gives
\begin{align*}
    \pdv{\Lag}{\psi} = \bar\psi\bqty{\gamma_\mu(i\p^\mu - g_v V^\mu) - (M-g_s\phi)}.
\end{align*}
Combining, we get our fourth equation
\begin{align}\label{eqn: eom,psi}
    i \p_\nu \bar\psi \gamma^\nu -\bar\psi\bqty{\gamma_\mu(i\p^\mu - g_v V^\mu) - (M-g_s\phi)} = 0.
\end{align}
In summary, here are the four equations of motion for QHD-I:
\begin{align*}
    &\p_\nu \p^\nu \phi + m_s\phi = g_s\bar{\psi}\psi,\tag{\ref{eqn: eom,sm}}\\
    &\p_\mu V^{\mu\nu} + m_\omega^2 V^\nu = g_v \bar\psi \gamma^\nu \psi,\tag{\ref{eqn: eom,vm}}\\
    &\bqty{\gamma_\mu(i\p^\mu -g_v V^\mu(x)) - (M-g_s\phi)} \psi = 0,\tag{\ref{eqn: eom,barpsi}}\\
    & i \p_\nu \bar\psi \gamma^\nu -\bar\psi\bqty{\gamma_\mu(i\p^\mu - g_v V^\mu) - (M-g_s\phi)} = 0. \tag{\ref{eqn: eom,psi}}
\end{align*}
The first three are given in \autocite{diener_2008}.

\section{Relativistic Mean Field Simplifications}

In \autocite{diener_2008} \S 8.3, we find that the equations of motion listed above are very difficult to solve in their current form. To make them more manageable, we approximate them with ``relativistic mean field'' (RMF) simplifications, where we take each field to be its ground state expectation value. For the meson fields, this simplification yields
\begin{align}
    \phi &\goesto \bra{\Phi} \phi \ket{\Phi} = \expval{\phi} \Def \phi_0, \\
    V_\mu & \goesto \bra{\Phi} V_\mu \ket{\Phi} = \expval{V_\mu} \Def \delta_{\mu 0} V_0,
\end{align}
where $\ket{\Phi}$ represents the ground state. These results arise from arguing that, in their ground states, $\phi$ and $V_\mu$ should be independent of space and time, as the system is bot uniform and stationary; therefore, $\phi_0$ and $V_0$ are constants. Furthermore, because the system is at rest and the baryon flux, $\bar\psi\gamma^i\psi$, is zero, the spatial components of the expected value of $V_\mu$, $\expval{V_\mu}$, must vanish \autocite{diener_2008}.

For the baryon field, a ``normal order'', i.e. normalized, expectation value must be taken, as because otherwise, the vacuum would be taken into account and the traditional expectation value would diverge. This ``normal ordered'' expectation value is denoted with a ``:''. Throughout the equations of motion, the 

\medskip
\JN{MORE HERE.}
% simplifications for baryon field
% derivation of energy momentum tensor and lagrangian for RMF
% derivation of energy density and pressure from EMT
% calculation of expectation values
\medskip

After evaluating these expectation values, we have
\begin{align}
    \phi_0 &= \frac{g_s}{m_s^2} \frac{\gamma}{2\pi^2}\int_0^{k_f} \dd{k} \frac{k^2 m^*}{\sqrt{k^2 + m^{*2}}}, \label{eqn: rmf,phi0} \\
    V_0 &= \frac{g_v}{m_\omega^2} \rho, \label{eqn: rmf,V0}
\end{align}
where once again, $m^* \Def M - g_s \phi$, the \emph{reduced mass}, and $\rho$ is the nucleon number density. If we assume spherical symmetry, a reasonable condition for the study of star-like systems, we get thee following expressions
\begin{align}
    \epsilon & = \frac{1}{2} m_s^2 \phi_0^2 + \frac{1}{2} m_\omega^2 V_0^2 + \frac{\gamma}{2\pi^2} \int_0^{k_f} \dd{k} k^2 \sqrt{k^2 + m^{*2}}, \label{eqn: rmf,eps}\\
    P & = -\frac{1}{2} m_s^2 \phi_0^2 + \frac{1}{2} m_\omega^2 V_0^2 + \frac{1}{3} \pqty{\frac{\gamma}{2\pi^2} \int_0^{k_f} \dd{k}\frac{k^4}{\sqrt{k^2 + m^{*2}}}}. \label{eqn: rmf,p}
\end{align}
In the next section, we will use these expressions to generate values for this equation of state.

\section{Numerical generation of the equation of state}

% We now wish to generate tabulated values of the equation of state using the equations for $\epsilon, P$ and $\phi_0$ at the end of the previous section. To do so, we first wish to find analytical forms for the integrals present in each of those equations, respectively. The aforementioned integrals are the following
% \begin{align}
%     \int_{0}^{k_f} \dd{k} \frac{k^2 m^*}{\sqrt{k^2 + m^{*2}}}, \quad\int_{0}^{k_f} \dd{k} \frac{k^4}{\sqrt{k^2 + m^{*2}}},
%     \quad\int_{0}^{k_f} \dd{k} k^2\sqrt{k^2 + m^{*2}}.
% \end{align}
% We integrate each of these using Mathematica, giving it the assumptions that $k_f > 0$ and that $\Re[m^*] > 0$. This gives these explicit, closed forms
% \begin{align}
%     \int_{0}^{k_f} \dd{k} \frac{k^2 m^*}{\sqrt{k^2 + m^{*2}}} &= \frac{1}{2} m^* \pqty{k_f \sqrt{k_f^2 + m^{*2}} - m^{*2}\ln\bqty{\frac{k_f + \sqrt{k_f+m^{*2}}}{k_f}}} \\
%     \int_{0}^{k_f} \dd{k} \frac{k^4}{\sqrt{k^2 + m^{*2}}} &= \frac{1}{8} \pqty{k_f(2k_f^2 - 3m^{*2}) - 3m^{*4}\ln\bqty{\frac{k_f + \sqrt{k_f+m^{*2}}}{k_f}}}\\
%     \quad\int_{0}^{k_f} \dd{k} k^2\sqrt{k^2 + m^{*2}} &= \frac{1}{8} \pqty{k_f \sqrt{k_f^2 + m^{*2}}\qty(2k_f^2 + m^{*2}) - m^{*4}\ln\bqty{\frac{k_f + \sqrt{k_f+m^{*2}}}{k_f}}}
% \end{align}

We now wish to generate tabulated values of the equation of state using the equations for $\epsilon, P$ and $\phi_0$ at the end of the previous section. Simply, this process requires looping through various values of $k_f$; at each iteration, we find the corresponding value of $\phi_0$ by using a root finding routine on \eqref{eqn: rmf,phi0}, as $m^*$ depends on $\phi_0$, then computing $V_0$ independently using \eqref{eqn: rmf,V0}, and then finally substituting those values into \eqref{eqn: rmf,eps} and \eqref{eqn: rmf,p} and storing those values. After creating a table of values for $\epsilon$ and $P$, we can verify the validity of the equation of state by solving the TOV equations and comparing the results of static solutions, mass-radius curves, and mass-pressure curves to other, previously calculated equations of state.

\medskip
\begin{itemize}
\JN{
    \item This section is rushed. Need better explanations of TOV equations, etc. in a different section; see Chapter 2, the TOV equations section. This can be referenced here.
    \item Talk about the parameter set for QHD-1.
}
\end{itemize}
\medskip

To begin, we choose a value of $k_f$ for the entirety of the iteration; for sake of example, we take $k_f = 1$. Then, we find the value of $\phi_0$ for that $k_f$ value. We define a function in Python for $\phi_0$ from \eqref{eqn: rmf,phi0}. 

\begin{lstlisting}
def f_phi0(phi0, kf):
    mstar = M - g_s * phi0
    def f(k): # integrand
        return k**2 * mstar/sqrt(k**2 + mstar**2)
    integral, err = quad(f,0,kf)
    return phi0 - (g_s/m_s**2) * (gamma/(2*pi**2)) * integral
\end{lstlisting}

While most of the function is straight forward, lines 3-5 may be cryptic without context. These lines are responsible for numerically computing the integral at the end of \eqref{eqn: rmf,phi0}, given the current $k_f$ value. \code{f(x)} is simply a function definition for the integrand of that integral, while line 5 uses the SciPy function \code{quad} to numerically integrate \code{f} from $0$ to $k_f$. \code{quad} returns a tuple containing both the numerical value of the integration and the bounds on that value's error, so we unpack it to get the value we want and call it \code{integral}. 

To determine the value of $\phi_0$, we must use a root finding routine because in \eqref{eqn: rmf,phi0}, $\phi_0$ appears on both sides of the equation.

\JN{Unfinished.} Within a tabulated equation of state, we need values for pressure from approximately $\SI{e-14}{}$ to about $\SI{e-1}{}$. When solving for a static solution using the TOV equations, we terminate the integration when the pressure drops below a certain threshold, which in our case is about $\SI{e-11}{}$, so we want values of the equation of state below that point in order to ensure that we calculate the solution correctly as it goes to zero. 