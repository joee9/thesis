\documentclass[fleqn,handout]{beamer}
    \usetheme{Boadilla}
    \usecolortheme{beaver}

% \usepackage[fleqn]{amsmath}
\usepackage[italicdiff]{physics}
\usepackage{siunitx}
\usepackage{amssymb,xparse}
	\usepackage[makeroom]{cancel} %% \cancelto{value}{expression}

\usepackage{graphicx}

\usepackage{dutchcal}  %changes \mathcal, \mathbcal for script r (E&M)
\newcommand{\Letter}[1]{\mathcal{#1}}
\newcommand{\Lag}{\Letter{L}}
\newcommand{\Sl}{\ell} % Lowercase script l 
\newcommand{\Def}{\equiv}
\newcommand{\goesto}{\quad\Rightarrow\quad}
\newcommand{\prop}{\propto}
\newcommand{\p}{\partial}

\title{Analysis of Equations of State for Neutron Star Modelling}
\author{Joseph Nyhan}
\date{April 5, 2022}
\institute{College of the Holy Cross}

\begin{document}

    \maketitle

    \begin{frame}{Outline}
        \begin{itemize}
            \item What is an equation of state (EoS)? \pause How do they fit into our model of a neutron star? \pause
            \item Using an Eos to make macroscopic predictions \pause
            \item Analysis and derivation of two EoSs: \pause
            \begin{itemize}
                \item QHD-I \pause
                \item NL3
            \end{itemize}
        \end{itemize}
    \end{frame}

    \begin{frame}{Equation of State (EoS)}
        \begin{itemize}
            \item A relationship between \textit{energy density} (denoted $\epsilon$) \pause and pressure (denoted $P$) \pause \begin{itemize}
                \item $\epsilon = \epsilon (P)$\pause $~~ \Leftrightarrow ~~  P = P(\epsilon)$
            \end{itemize}
            \item Encodes the fundamental interparticle interactions within a neutron star \pause
            \item True EoS within a neutron star is unknown; \pause multitude of candidates, each based on a slightly different model
        \end{itemize}
    \end{frame}

    \begin{frame}{Using an Equation of State to Make Predictions}
        \begin{itemize}
            \item TOV equations \pause
            \item Mass rad
        \end{itemize}
    \end{frame}

    \begin{frame}{TOV Equations}
        \begin{itemize}
            \item The Tolman-Oppenheimer-Volkoff Equations
        \end{itemize}
    \end{frame}

\end{document}

