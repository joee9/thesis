\chapter{Advanced QHD Parameter Sets}

\section{Equations of Motion}

The three equations of motion for the meson fields reduce to, as shown in \autocite[p. 79]{diener_2008}
\begin{align}
    \phi_0 & = \frac{g_s}{m_s^2}\bqty{\frac{1}{\pi^2} \pqty{\int_0^{k_p} \dd{k} \frac{k^2 m^*}{\sqrt{k^2 + m^{*2}}} + \int_0^{k_n} \dd{k} \frac{k^2 m^*}{\sqrt{k^2 + m^{*2}}}} -\frac{\kappa}{2}(g_s\phi_0)^2 -\frac{\lambda}{6}(g_s\phi_0)^3 }\nonumber\\
    V_0 & = \frac{g_v}{m_s^2} \bqty{\rho_p + \rho_n - \frac{\zeta}{6}(g_v V_0)^3 - 2 \Lambda_v (g_vV_0)(g_\rho b_0)^2}\nonumber\\
    b_0 & = \frac{g_\rho}{m_\rho^2} \bqty{\frac{1}{2}(\rho_p - \rho_n) - 2\Lambda_v(g_vV_0)^2(g_\rho b_0)}
\end{align}

\section{Equilibrium Conditions}

The number densities of baryons must stay constant; thus, we have
\begin{align}
    \rho = \rho_n + \rho_p,
\end{align}
where $\rho_n$ is the number density of the first species, neutrons, while $\rho_p$ is the number density of the second species, protons. It is important to note now that we use the relation given on \autocite[p. 90]{diener_2008}
\begin{align*}
    \rho_x = \frac{k_x^3}{3\pi^2} \quad\Longleftrightarrow\quad k_x = \pi^2 (3\pi^2\rho_x)^{-2/3}
\end{align*}
to relate the $x$th fermi-momenta with its corresponding number density.

For beta-equilibrium to be satisfied, we must have
\begin{align}
    \mu_n = \mu_p + \mu_e,
\end{align}
where $\mu_x$ is the $x$th \textit{chemical potential.} Using a handy result from \autocite[p. 90]{diener_2008}, we can rewrite this more generally as
\begin{align}
    \sqrt{k_n^2 + m^{*2}} = \sqrt{k_p^2 + m^{*2}} + g_\rho b_0 + \sqrt{k_e^2 + m_e^2}.
\end{align}

Furthermore, within neutron stars, muon production is probable and often favorable, per \autocite[p. 90]{diener_2008}. Therefore, we have
\begin{align}
    \mu_\mu = \mu_e,\quad \mu_e  = \sqrt{k_e^2 + m_e^2}, \quad \mu_\mu  = \sqrt{k_\mu^2 + m_\mu^2}.
\end{align}

Because charge must be conserved, we need to have
\begin{align}
    \rho_p = \rho_e + \rho_\mu \goesto k_p = \pqty{k_e^3 + k_\mu^3}^{1/3}
\end{align}

\section{Relativistic Mean Field Simplifications}
\section{Numerical generation of the equation of state}
\section{Analysis}