\newcommand{\B}{{\boldsymbol b}}

\chapter{Advanced QHD Parameter Sets}

\section{Introduction}

\section{Equations of Motion}


In this section, we work with a more general Lagrange density for QHD:
\begin{align}
    \Lag & = \psi \bqty{\gamma^\mu\qty(i\p_\mu -g_vV_\mu - \frac{g_\rho}{2}{\boldsymbol\tau}\cdot {\boldsymbol b}_\mu)-(M-g_s\phi)}\psi\nonumber\\
    & \quad + \frac{1}{2}\p_\mu \p^\mu \phi - \frac{1}{2}m_s^2\phi^2 - \frac{\kappa}{3!}(g_s\phi)^3 - \frac{\lambda}{4!}(g_s\phi)^4 \nonumber\\
    & \quad - \frac{1}{4} V^{\mu\nu}V_{\mu\nu} + \frac{1}{2} m_\omega^2 V^\mu V_\mu + \frac{\zeta}{4!}(g_v^2 V^\mu V_\mu)^2 \nonumber\\
    & \quad - \frac{1}{4} \B^{\mu\nu} \B_{\mu\nu} + \frac{1}{2}m_\rho^2 \B^\mu \cdot \B_\mu + \Lambda_v (g_v^2 V^\mu V_\mu)(g_\rho^2 \B^\mu \B_\mu),
\end{align}
where
\begin{align*}
    V_{\mu\nu} \Def \p_\mu V_\nu - \p_\nu V_\mu, \quad
    \B_{\mu\nu} \Def \p_\mu \B_\nu - \p_\nu \B_\mu.
\end{align*}


The three equations of motion for the meson fields reduce to, as shown in \autocite[p. 79]{diener_2008}
\begin{align}
    \phi_0 & = \frac{g_s}{m_s^2}\bqty{\frac{1}{\pi^2} \pqty{\int_0^{k_p} \dd{k} \frac{k^2 m^*}{\sqrt{k^2 + m^{*2}}} + \int_0^{k_n} \dd{k} \frac{k^2 m^*}{\sqrt{k^2 + m^{*2}}}} -\frac{\kappa}{2}(g_s\phi_0)^2 -\frac{\lambda}{6}(g_s\phi_0)^3 }\nonumber\\
    V_0 & = \frac{g_v}{m_s^2} \bqty{\rho_p + \rho_n - \frac{\zeta}{6}(g_v V_0)^3 - 2 \Lambda_v (g_vV_0)(g_\rho b_0)^2}\nonumber\\
    b_0 & = \frac{g_\rho}{m_\rho^2} \bqty{\frac{1}{2}(\rho_p - \rho_n) - 2\Lambda_v(g_vV_0)^2(g_\rho b_0)}
\end{align}

\section{Equilibrium Conditions}

The number densities of baryons must stay constant; thus, we have
\begin{align}
    \rho = \rho_n + \rho_p,
\end{align}
where $\rho_n$ is the number density of the first species, neutrons, while $\rho_p$ is the number density of the second species, protons. It is important to note now that we use the relation given on \autocite[p. 90]{diener_2008}
\begin{align*}
    \rho_x = \frac{k_x^3}{3\pi^2} \quad\Longleftrightarrow\quad k_x = (3\pi^2\rho_x)^{1/3}.
\end{align*}
to relate the $x$th fermi-momenta with its corresponding number density.

For beta-equilibrium to be satisfied, we must have
\begin{align}
    \mu_n = \mu_p + \mu_e,
\end{align}
where $\mu_x$ is the $x$th \textit{chemical potential.} Using a handy result from \autocite[p. 90]{diener_2008}, we can rewrite this more generally as
\begin{align}
    \sqrt{k_n^2 + m^{*2}} = \sqrt{k_p^2 + m^{*2}} + g_\rho b_0 + \sqrt{k_e^2 + m_e^2}.
\end{align}

Furthermore, within neutron stars, muon production is probable and often favorable, per \autocite[p. 90]{diener_2008}. Therefore, we have
\begin{align}
    \mu_\mu = \mu_e,\quad \mu_e  = \sqrt{k_e^2 + m_e^2}, \quad \mu_\mu  = \sqrt{k_\mu^2 + m_\mu^2}.
\end{align}

Because charge must be conserved, we need to have
\begin{align}
    \rho_p = \rho_e + \rho_\mu \goesto k_p = \pqty{k_e^3 + k_\mu^3}^{1/3}
\end{align}

\section{Relativistic Mean Field Simplifications}

After evaluating the expectation values, $\epsilon$ and $P$ take the forms
\begin{align}
    \epsilon = & +\frac{1}{2}m_s^2\phi_0^2 + \frac{\kappa}{3!} (g_s\phi_0)^3 + \frac{\lambda}{4!}(g_s\phi_0)^4 - \frac 12 m_\omega^2V_0^2 - \frac{\zeta}{4!}(g_vV_0)^4 \nonumber\\
    & - \frac 12 m_\rho^2 b_0^2 - \Lambda_v(g_vV_0)^2(g_\rho b_0)^2 + g_vV_0(\rho_n + \rho_p) + \frac{1}{2} (\rho_p-\rho_n) \nonumber\\
    & + \frac{1}{\pi^2}\bqty{\int_0^{k_p} \dd{k} k^2 \sqrt{k^2 + m^{*2}} + \int_0^{k_n} \dd{k} k^2 \sqrt{k^2 + m^{*2}}}\\
    P = & -\frac{1}{2}m_s^2\phi_0^2 - \frac{\kappa}{3!} (g_s\phi_0)^3 - \frac{\lambda}{4!}(g_s\phi_0)^4 + \frac 12 m_\omega^2V_0^2 + \frac{\zeta}{4!}(g_vV_0)^4 \nonumber\\
    & + \frac 12 m_\rho^2 b_0^2 + \Lambda_v(g_vV_0)^2(g_\rho b_0)^2 \nonumber\\
    & + \frac{1}{3\pi^2}\bqty{\int_0^{k_p} \dd{k} \frac{k^4}{\sqrt{k^2 + m^{*2}}} + \int_0^{k_n} \dd{k} \frac{k^4}{\sqrt{k^2 + m^{*2}}}}
\end{align}

\section{Numerical generation of the equation of state}



\section{Analysis}