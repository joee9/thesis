\documentclass[handout]{beamer}
% \documentclass[]{beamer}
    \usetheme{Boadilla}
    \usecolortheme{beaver}

% \usepackage[fleqn]{amsmath}
\usepackage[italicdiff]{physics}
\usepackage{siunitx}
\usepackage{amssymb,xparse}
	\usepackage[makeroom]{cancel} %% \cancelto{value}{expression}
\usepackage{caption}
\usepackage{subcaption}

\usepackage[backend=biber]{biblatex}
    \addbibresource{refs.bib}

\usepackage{graphicx}

\usepackage{courier}
\newcommand{\code}[1]{{\fontfamily{pcr}\selectfont #1}}

\usepackage{dutchcal}  %changes \mathcal, \mathbcal for script r (E&M)
\newcommand{\Letter}[1]{\mathcal{#1}}
\newcommand{\Lag}{\Letter{L}}
\newcommand{\Sl}{\ell} % Lowercase script l 
\newcommand{\Def}{\equiv}
\newcommand{\goesto}{\quad\Rightarrow\quad}
\newcommand{\prop}{\propto}
\newcommand{\p}{\partial}

\title[Analysis of Equations of State]{Analysis of Equations of State in Neutron Star Modeling and Simulation}
\author{Joseph Nyhan}
\date{6 May 2022}
\institute[CoHC]{College of the Holy Cross}

\begin{document}

    \maketitle

    \begin{frame}{Outline}
        \pause
        \begin{itemize}
            \item What is a neutron star? \pause
            \item What is an equation of state (EoS)? \pause How does it fit into our model of a neutron star? \pause
            \item How can we use an EoS to make macroscopic predictions about neutron stars? \pause
            \item Temporal simulations of neutron stars and equations of state
            \item A derivation of an EoS and its predictions: \pause
            \begin{itemize}
                \item Quantum Hadrodynamics and the QHD-I parameter set
            \end{itemize}
        \end{itemize}
    \end{frame}

    \begin{frame}{Neutron Stars}
        \pause
        \begin{itemize}
            \item Dense core left behind after a supernovae explosion
            \item Made mostly of neutrons, protons, and electrons\pause ; overall, is neutral \pause
            \item Radius: $\sim\SI{10}{km}$\pause ; Mass: $\sim\SI{1}{\odot}$. \pause
            \item Approximately the density of atomic nuclei ($\sim\SI{e17}{kg/m^3}$)
            \item Core held together by intense gravitational attraction \pause
            \begin{itemize}
                \item Gravitational acceleration on Earth's Surface: $\approx \SI{10}{m/s^2}$ \pause
                \item Neutron star: $\approx \SI{e12}{m/s^2}$ \pause (escape velocity $\sim \SI{100000}{km/s} = $ \pause $c/3$)\pause
            \end{itemize}
            \item Why are they interesting? \pause \begin{itemize}
                \item Smallest, densest observed stellar objects \pause
                \item Exotic physics
            \end{itemize}
        \end{itemize}
    \end{frame}

    \begin{frame}{Equation of State (EoS)}
        What is an equation of state? \pause
        \begin{itemize}
            \item A relationship between \textit{energy density} (denoted $\epsilon$) \pause and pressure (denoted $P$) \pause \begin{itemize}
                \item $\epsilon = \epsilon (P)$\pause $~~ \Leftrightarrow ~~  P = P(\epsilon)$ \pause
            \end{itemize}
            \item Encodes the fundamental interparticle interactions within a neutron star \pause
            \item True EoS within a neutron star is unknown; \pause multitude of candidates, each based on a slightly different model and fit of empirical data \pause \begin{itemize}
                \item Models can be very complicated\pause ; often simplifications must be made to be solved practically \pause
            \end{itemize}
            \item Often a tabulated list of $P$ and $\epsilon$ values; \pause however, in simulation work, analytical fits may be required
        \end{itemize}
    \end{frame}

    % \begin{frame}{Using an Equation of State to Make Predictions}
    %     \begin{itemize}
    %         \item TOV equations \pause
    %         \item Mass rad
    %     \end{itemize}
    % \end{frame}

    \begin{frame}{Using an EoS to Make Predictions}
        We want a way to understand the effects of an EoS on the observable properties of a star
        \begin{itemize}
            \item e.g. total mass, total radius
        \end{itemize}
        We create \textit{static solutions}; ``images'' of neutron star
        \begin{itemize}
            \item solve the Tolman-Oppenheimer-Volkoff (TOV) equations
            \item extract information about maximum max and radius allowed by the EoS
        \end{itemize}
    \end{frame}

    \begin{frame}{The Tolman-Oppenheimer-Volkoff (TOV) Equations}
        \pause Used to describe a static (time independent) spherically symmetric star. \pause Given by
        \begin{align*}
            \dv{m}{r} = 4\pi r^2 \epsilon, \quad \dv{P}{r} = -\frac{(4\pi r^3 P + m)(\epsilon + P)}{r^2 (1-2m/r)}.
        \end{align*}
        where $\epsilon$ is energy density, $P$ is pressure, and $m$ is ``mass.'' \pause Use EoS to determine $\epsilon~$\pause $=\epsilon(P)$. \pause 
        \medskip
        
        \begin{itemize}
            \item Initial conditions: \pause
            \[m(r=0) = 0, \quad P(r=0) \Def P_0 = \text{const.}\]\pause
            Each solution is uniquely defined by $P_0$, the \textit{central pressure}.
            \item Outer conditions: Let $R,M$ to be the total radius and total mass of the star, respectively. Defined by:
            \[P(R) = 0, \quad M = m(R).\]
        \end{itemize}
    \end{frame}

    \begin{frame}{TOV Equations: Computing a Solution}
        \begin{align*}
            \dv{m}{r} = 4\pi r^2 \epsilon, \quad \dv{P}{r} = -\frac{(4\pi r^3 P + m)(\epsilon + P)}{r^2 (1-2m/r)}.
        \end{align*}
        \begin{itemize}
            \item Specify a central pressure $P(r=0) = P_0$
            \item Begin at very small $r \approx 0$; (\SI{e-8}{})
            \item Use a numerical integration technique
            \begin{itemize}
                \item The Runge-Kutta 4 Algorithm
                \item In practice, use Scipy \code{solve\_ivp}; faster due to optimized step size
            \end{itemize}
            \item Integrate outwards until $P = 0$; use to define $R$, calculate $M$
            \item Store curves for $m(r), P(r)$, use to calculate $\epsilon(r)$
        \end{itemize}
    \end{frame}

    \begin{frame}{Static Solution: Example}
        \pause
        Use an EoS called ``SLy'' from \autocite{SLy_2004}. A realistic equation of state from an analytical fit of empirical neutron star data.

        \begin{figure}[h!]
            \centering
            \begin{subfigure}{.5\textwidth}
                \includegraphics[width = \textwidth]{SLy_P,p0_0.0001.pdf}
            \end{subfigure}%
            \begin{subfigure}{.5\textwidth}
                \includegraphics[width = \textwidth]{SLy_rho,p0_0.0001.pdf}
            \end{subfigure}
            \caption[]{Example static solution for $P_0 = \SI{e-4}{}$ for an EoS called ``SLy.''}
        \end{figure}

    \end{frame}

    \begin{frame}{Static Solutions: $M(R)$ and $M(P_0)$ diagrams}
        \pause
        Single solutions don't tell much about star as a whole; instead, look at trends over lots of solutions
        \begin{enumerate}
            \item Create static solutions for a range of $P_0$ values: \pause $P_0 \in [\SI{e-6}{}, \SI{e-1}{}].$\pause
            \item Calculate the total mass $M$ and total radius $R$ for each value of $P_0$ \pause
            \item Plot $M(R)$ and $M(P_0)$\pause
        \end{enumerate}

        \begin{figure}[h!]
            \centering
            \begin{subfigure}{.5\textwidth}
                \includegraphics[width = \textwidth]{r_analysis,SLy.pdf}
            \end{subfigure}%
            \begin{subfigure}{.5\textwidth}
                \includegraphics[width = \textwidth]{p0_analysis,SLy.pdf}
            \end{subfigure}
            \caption[]{Example curves for EoS ``SLy.'' $\SI{1}{\odot} = \SI{1.989e+30}{kg}$ (solar mass)}
        \end{figure}
    \end{frame}

    \begin{frame}{Critical Values of $P$, $R$, and $M$}
        \vspace{-10pt}
        \begin{figure}[h!]
            \centering
            \begin{subfigure}{.5\textwidth}
                \includegraphics[width = \textwidth]{r_analysis,SLy.pdf}
            \end{subfigure}%
            \begin{subfigure}{.5\textwidth}
                \includegraphics[width = \textwidth]{p0_analysis,SLy.pdf}
            \end{subfigure}
        \end{figure} \pause
        Three important values: \pause \textit{critical pressure, critical mass,} and \textit{critical radius}. \pause 
        \begin{itemize}
            \item Determined by ``peaks'' of graph\pause ; calculated using an optimization routine \pause
            \item Maximum mass and radius predicted by EoS \pause
            \item Largest ``stable'' pressure \pause
            \item For SLy, $M_\text{max} = \SI{2.05}{\odot}$, $R_\text{max} = \SI{9.93}{km}$, and $P_\text{crit} = \SI{6.59e-3}{}$.
        \end{itemize}

    \end{frame}

    % \begin{frame}{Predictions of Analytical Fit Equations of State}

    % \end{frame}

    \begin{frame}{Temporal Simulations of Neutron Stars: Background}
        \begin{itemize}
            \item Neutron star as a spherically symmetric hydrodynamical system
            \item Three \textit{primitive} variables: pressure $P$, energy density $\epsilon$, and velocity $v$
            \item $\epsilon$ and $P$ related by an EoS
            \item Define \textit{conservative} variables $\Pi, \Phi$ in terms of primitive variables
            \[\Pi = \frac{\epsilon + P}{1-v} - P, \quad \Phi = \frac{\epsilon + P}{1+v} - P.\]
            \item $\Pi, \Phi$ obey a conservation equation
            \[\p_t \vec{u} = -\frac{1}{r^2}\p_r\pqty{r^2\frac{\alpha}{a}\vec{f}^{(1)}} - \p_r\qty(\frac{\alpha}{a}\vec{f}^{(2)})+\vec{s}, \quad \vec{u} = \mqty[\Pi\\\Phi],\]
            where $a,\alpha$ are the \textit{gravity} variables.
            \item $\vec{f}^{(1)} = \vec{f}^{(1)} (\vec{u}, v),~~ \vec{f}^{(2)} = \vec{f}^{(2)}(P),~~ \vec{s} = \vec{s}~(\vec{u},P,\epsilon,v,a,\alpha)$.
            \item Separate evolution equations for $a,\alpha$
        \end{itemize}
    \end{frame}


    \begin{frame}{Temporal Simulations of Neutron Stars: Background}
        \begin{itemize}
        \item Evolve a set of discrete spatial gridpoints $\to$ advanced numerical techniques
        \begin{itemize}
            \item Finite differencing (for spatial derivatives) and the method of lines
            \item High-resolution shock-capturing methods
            \item Evolve through time using numerical integration (Runge-Kutta 3, Modified Euler's Method)
        \end{itemize}
        \item Use EoSs that are analytical fits for numerical stability and root-finding abilities \begin{itemize}
            \item Determine $P,\epsilon,v$ numerically from $\Pi, \Phi$; need to be able to differentiate (e.g. Newton-Raphson Method)
            \item Extensive studies of realistic, analytical EoSs from \autocite{SLy_2004,BSk_2013}; SLy family
        \end{itemize}
        \end{itemize}
    \end{frame}

    \begin{frame}{Temporal Simulations of Neutron Stars}

        \begin{itemize}
            \item Use static solutions as initial data for temporal simulation
            \begin{itemize}
                \item Above \textit{critical pressure}: unstable; below: stable
            \end{itemize}
            \item Stable solutions exhibit \textit{radial oscillations} \begin{itemize}
                \item Evolve out to large $t$ and perform a Fourier transform
            \end{itemize}
        \end{itemize}
            \vspace{-10pt}
        \begin{figure}[h!]
            \centering
            \begin{subfigure}{.5\textwidth}
                \includegraphics[width = \textwidth]{13-P,ringdown.pdf}
            \end{subfigure}%
            \begin{subfigure}{.5\textwidth}
                \includegraphics[width = \textwidth]{13-P,ringdown,fft.pdf}
            \end{subfigure}
            \vspace{-10pt}
            \caption[]{Plots of $P(r=0)$ for EoS ``SLy'' and initial $P_0 = \SI{7e-4}{GeV^4}$. Colored lines on FFT plot represent predicted frequencies.}
        \end{figure}
        \vspace{-15pt}
        \begin{itemize}
        \item Radial oscillations differ by EoS; they could soon be measurable!
        \end{itemize}

        % talk about temporal simulation 
        % static solutions as "initial data" for temporal simulation
        % use of analytical fits for equations of state
        % SLy and other realistic equations of state       
    \end{frame}

    % \begin{frame}{Application: Temporal Simulations of Neutron Stars}
    %     % radial oscillations over time; fourier transform
    %     % possible measurement in near future; another observable property
    %     % vary with equation of state; more insight into which EoS may be "correct"
    % \end{frame}

    \begin{frame}{Nearing Publication: }
        Only if necessary for time!
    \end{frame}

    \begin{frame}{Computing an EoS: Quantum Hadrodynamics}
        \pause
        A theory of the quantum mechanical, interparticle interactions within a neutron star. \pause
        \begin{itemize}
            \item Formulation of nuclear interactions between \textit{baryons} by the exchange of \textit{mesons} \pause \begin{itemize}
                \item \textit{baryons} are particles containing three quarks (e.g. protons, neutrons) \pause
                \item \textit{mesons} are quark/anti-quark pairs\pause
            \end{itemize}
            \item Requires experimental input for constraint; \pause implemented using \textit{coupling constants} \pause \begin{itemize}
                \item Models the strength of the interactions between particles
                \item Multiple \textit{parameter sets} have been developed by fitting observed nuclear properties of nuclear matter \pause
            \end{itemize}
            \item Considered quite complicated to solve\pause ; we introduce some simplifications in the QHD-I model
        \end{itemize}
    \end{frame}

    % \begin{frame}{Quantum Hadrodynamics}
    %     \pause A theory of the interactions between the subatomic particles in the core of a neutron star
    %     \pause
    %     \begin{itemize}
    %         \item The protons and neutrons influence one another (e.g. apply forces); \pause mediated by exchanging a particle \pause (called a \textit{meson}) \pause
    %         \item Describe these interactions mathematically; \pause the Lagrangian $\Lag$ \pause
    %     \end{itemize}
    %     Model requires input from actual experiments \pause
    %     \begin{itemize}
    %         \item Extreme conditions in core of a neutron star are not reproducible \pause (e.g. pressures are too high, etc.) \pause
    %     \end{itemize}
    %     Equations of Quantum Hadrodynamics are considered very difficult to solve; \pause introduce simplifications in the QHD-I model
    % \end{frame}

    \begin{frame}{Quantum Hadrodynamics I (QHD-I)}
        We form the Lagrange Density for QHD-I: \pause
        \begin{align*}
            \Lag & = \bar{\psi} \bqty{\gamma_\mu(i\p^\mu -g_v V^\mu) - (M-g_\phi\phi)} \psi \nonumber\\
            & \quad + \frac{1}{2} \pqty{\p_\mu \phi \p^\mu \phi - m_\phi^2 \phi^2} - \frac{1}{4} V_{\mu\nu}V^{\mu\nu} + \frac{1}{2} m_v^2 V_\mu V^\mu,
        \end{align*}\pause
        where $V_{\mu\nu}\Def\p_\mu V_\nu - \p_\nu V_\mu$, $\p_\mu \Def \pdv*{x^\mu}.$ The fields:
        \begin{itemize}
            \item Baryon field (protons and neutrons) $\psi(x^\mu)$, with mass $M$ \pause
            \item Scalar meson field: $\phi(x^\mu)$, with mass $m_\phi$
            \item Vector meson field: $V^\mu(x^\mu)$, with mass $m_v$
            \item Experimental coupling constants: $g_v$ and $g_\phi$ \pause
        \end{itemize}
        From $\Lag$, we can determine $\epsilon$ and $P$, the EoS we desire.
    \end{frame}

    \begin{frame}{QHD-I: Derivation of Equations of Motion}
        \begin{align*}
            \Lag & = \bar{\psi} \bqty{\gamma_\mu(i\p^\mu -g_v V^\mu) - (M-g_\phi\phi)} \psi \nonumber\\
            & \quad + \frac{1}{2} \pqty{\p_\mu \phi \p^\mu \phi - m_\phi^2 \phi^2} - \frac{1}{4} V_{\mu\nu}V^{\mu\nu} + \frac{1}{2} m_v^2 V_\mu V^\mu,
        \end{align*}\pause
        Applying the Euler-Lagrange equations for $\Lag$ over a classical field
        \[\p_\nu\qty(\pdv{\Lag}{(\p_\nu \varphi_\alpha)}) - \pdv{\Lag}{\varphi_\alpha} = 0,\]
        $\varphi_\alpha \in \Bqty{\phi, V^\mu, \psi}$, we obtain the equations of motion:
        \begin{align*}
            &\p_\nu \p^\nu \phi + m_s^2\phi = g_s\bar{\psi}\psi,\\
            &\p_\mu V^{\mu\nu} + m_\omega^2 V^\nu = g_v \bar\psi \gamma^\nu \psi,\\
            &\bqty{\gamma_\mu(i\p^\mu -g_v V^\mu) - (M-g_s\phi)} \psi = 0,\\
        \end{align*}
    \end{frame}

    \begin{frame}{QHD-I: RMF Simplifications}
        We introduce the \textit{Relativistic Mean Field} (RMF) simplifications. \pause We treat the interactions (exchange of mesons) as their average values: \pause
        \[\phi \to \expval{\phi} = \phi_0, \quad V_\mu \to \expval{V_\mu} = V_0, \quad \bar\psi\psi \to \expval{\bar\psi\psi}, \quad \bar\psi\gamma^\mu\psi \to \expval{\bar\psi \gamma^0\psi},\]
        where $\phi_0$ and $V_0$ are constants. \pause This allows us to simplify $\Lag$ considerably: \pause
        \begin{align*}
            \Lag_\text{RMF} = \bar\psi \bqty{i\gamma_\mu\p^\mu - g_v \gamma_0 V_0 - (M-g_s\phi_0)} \psi - \frac{1}{2} m_s^2 \phi_0^2 + \frac{1}{2} m_\omega^2 V_0^2,
        \end{align*} \pause
        Applying the same simplifications to the equations of motions gives:
        \begin{align*}
            & m_s^2 \phi_0^2 = g_s\expval{\bar\psi\psi}\\
            & m_\omega^2 V_0 = g_v\expval{\bar\psi\gamma^0\psi}\\
            & \bqty{i\gamma_\mu\p^\mu - g_v\gamma_0 V_0 - (M - g_s\phi_0)}\psi = 0\\
        \end{align*}

        % Determining $\phi_0$, $V_0$, $\epsilon$, and $P$: \pause $\quad\varphi_\alpha \in \Bqty{\phi_0, V_0, \psi}$
        % \[\p_\nu\qty(\pdv{\Lag}{(\p_\nu \varphi_\alpha)}) - \pdv{\Lag}{\varphi_\alpha} = 0, \quad
        % T^{\mu\nu} = \pdv{\Lag}{(\p_\mu \varphi_\alpha)}\p^\nu \varphi_\alpha - \Lag \eta^{\mu\nu}.\] \pause
        % $$\epsilon = \expval{T^{00}}, \quad P = \expval{T^{ii}}$$
    \end{frame}

    \begin{frame}{QHD-I: Closed forms for $\epsilon$ and $P$}
        We can now find closed forms of $\epsilon$ and $P$. From \autocite{diener_2008}, we have
        \[\epsilon = \expval{T^{00}}, \quad P = \frac{1}{3}\expval{T^{ii}},\]
        where $T^{\mu\nu}$ is the energy momentum tensor, given by
        \vspace{-5pt}
        \begin{align*}
            T^{\mu\nu} = \pdv{\Lag}{(\p_\mu \varphi_\alpha)}\p^\nu \varphi_\alpha - \Lag \eta^{\mu\nu}.
        \end{align*}
        \vspace{-5pt}
        Using $\Lag_\text{RMF}$
        \vspace{-5pt}
        \begin{align*}
            T^{\mu\nu}_\text{RMF} & =i\bar\psi\gamma^\mu \p^\nu \psi - \eta^{\mu\nu} \pqty{- \frac{1}{2} m_s^2 \phi_0^2 + \frac{1}{2} m_\omega^2 V_0^2}.
        \end{align*}
        \vspace{-5pt}
        This gives
        \vspace{-15pt}
        \begin{align*}
            \epsilon & = \expval{i\bar\psi\gamma^0 \p^0 \psi} + \frac{1}{2} m_s^2 \phi_0^2 - \frac{1}{2} m_\omega^2 V_0^2,\\
            P & = \expval{i\bar\psi\gamma^i \p^i \psi}  - \frac{1}{2} m_s^2 \phi_0^2 + \frac{1}{2} m_\omega^2 V_0^2.
        \end{align*}
        The above expectation values are non-trivial and are derived in \autocite{diener_2008}.
    \end{frame}

    \begin{frame}{QHD-I: Resulting Equations}
        From above, we obtain the following equations: \pause
        \begin{align*}
            \phi_0 &= \frac{g_\phi}{m_\phi^2} \frac{1}{\pi^2} \int_0^{k_f} \dd{k} \frac{(M-g_\phi \phi_0) k^2 }{\sqrt{k^2 + (M-g_\phi \phi_0)}},  \\
            V_0 &= \frac{g_v}{m_v^2} \frac{k_f^3}{3\pi^2}, \\
            \epsilon & = \frac{1}{2} m_\phi^2 \phi_0^2 + \frac{1}{2} m_v^2 V_0^2 + \frac{1}{\pi^2} \int_0^{k_f} \dd{k} k^2 \sqrt{k^2 + m^{*2}},\\
            P & = -\frac{1}{2} m_\phi^2 \phi_0^2 + \frac{1}{2} m_v^2 V_0^2 + \frac{1}{3} \pqty{\frac{1}{\pi^2} \int_0^{k_f} \dd{k}\frac{k^4}{\sqrt{k^2 + m^{*2}}}}.
        \end{align*}
        where $m^* = (M-g_\phi \phi)$, the \textit{reduced mass}. $k_f$, the Fermi wavenumber, is a free parameter. 
    \end{frame}

    \begin{frame}{Resulting Equations}
        \pause Goal: \pause create a list of values that show us $\epsilon(P)$; \pause each value of $k_f$ gives us a different $\epsilon$ and $P$. \pause

        \medskip
        To produce the EoS: \pause (repeat the following) \pause
        \begin{itemize}
            \item Choose a $k_f$ value \pause
            \item calculate $\phi_0$ and $V_0$\pause; use \textit{rootfinding} for $\phi_0$\pause
            \item Using those values, calculate $P$ and $\epsilon$ and store in a table
        \end{itemize}
        We loop through $k_f$ values until we have a large range of $P$ values \[P\in[\SI{e-20}{GeV^4}, \SI{e-1}{GeV^4}].\]
    \end{frame}

    % \begin{frame}{Producing the Equation of State}

    %     Steps: \pause
    %         \begin{align*}
    %             \phi_0 = f(k_f, \pause \phi_0), \pause \quad
    %             V_0 = \frac{g_v}{m_v^2} \frac{k_f^3}{3\pi^2} 
    %         \end{align*}
    %         \item Using those values, calculate $P$ and $\epsilon$ and store in a table
    %     \end{itemize}
    % \end{frame}
    
    \begin{frame}{$M(R)$ and $M(P_0)$ Curves for QHD-I}
        We use the tabulated values of $P$ and $\epsilon$ to solve the TOV equations: \pause
        \begin{figure}[h!]
            \centering
            \begin{subfigure}{.5\textwidth}
                \includegraphics[width = \textwidth]{../paper/images/qhd1/r_analysis.pdf}
            \end{subfigure}%
            \begin{subfigure}{.5\textwidth}
                \includegraphics[width = \textwidth]{../paper/images/qhd1/p0_analysis.pdf}
            \end{subfigure}
            \caption[]{$M(R)$ and $M(P_0)$ curves for QHD-I EoS.}
        \end{figure}\pause
        \vspace{-3pt}
        These curves give \[M_\text{max} = \SI{6.1}{\odot}, \quad R_\text{max} = \SI{42.1}{km}, \quad P_\text{crit} = \SI{1.98e-4}{}.\]
    \end{frame}

    \begin{frame}{Conclusion}
        \begin{itemize}
            \item An equation of state is a relationship between energy density and pressure within a neutron star \pause
            \item We use the TOV equations to predict the maximum mass and radius that a given EoS will produce \pause
            \item Within a temporal simulation of a neutron star, the static solutions from the TOV equations are used as initial data \begin{itemize}
                \item Can predict radial oscillation frequencies of neutron stars, which could soon be measurable
            \end{itemize}
            \item We use the QHD-I parameter set and RMF simplifications to solve a system of equations and generate an equation of state
        \end{itemize}
    \end{frame}

    \begin{frame}[allowframebreaks]{References}
        \nocite{*}
        \printbibliography
    \end{frame}


\end{document}
